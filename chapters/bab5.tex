% BAB V - PENUTUP

\subsection{Kesimpulan}

Berdasarkan seluruh rangkaian tahapan pengembangan sistem yang dimulai dari analisis kebutuhan hingga tahap implementasi, dapat disimpulkan bahwa aplikasi ``GOPOS (\textit{Go-Lang POS Assistant})'' telah berhasil diwujudkan sebagai asisten virtual cerdas yang handal bagi layanan PT Pos Indonesia. Penggunaan bahasa pemrograman Go yang dikombinasikan dengan arsitektur \textit{serverless} dan basis data MongoDB Atlas terbukti mampu menghasilkan sistem dengan performa tinggi yang dapat menangani data riwayat percakapan secara dinamis. Keberhasilan sistem ini juga didukung oleh integrasi model Gemini Pro AI yang mampu memberikan respon bahasa natural secara kontekstual, baik untuk kebutuhan informasi tarif pengiriman maupun layanan umum lainnya melalui platform web dan WhatsApp.

Implementasi antarmuka pada sisi frontend yang mencakup halaman beranda, \textit{dashboard} admin, dan fitur chat AI telah berfungsi secara optimal dengan tingkat responsivitas yang baik pada berbagai perangkat. Dari sisi validasi teknis, pengujian logika melalui metode \textit{White Box Testing} menggunakan \textit{unit testing} pada bahasa Go menunjukkan bahwa alur data internal, struktur koleksi pada database, serta fungsi normalisasi teks telah berjalan dengan akurat sesuai dengan spesifikasi yang dirancang. Secara teknis, sistem ini telah memenuhi standar kelayakan operasional tanpa adanya kesalahan logika krusial pada modul-modul utamanya, sehingga siap digunakan untuk meningkatkan efisiensi layanan pelanggan.

\subsection{Saran}

Meskipun sistem GOPOS saat ini telah berjalan dengan stabil, terdapat beberapa peluang pengembangan strategis yang dapat diimplementasikan untuk meningkatkan kapabilitas asisten virtual ini di masa depan. Pengembangan selanjutnya disarankan untuk mengintegrasikan sistem secara langsung dengan API pelacakan kiriman (\textit{tracking}) resmi dari PT Pos Indonesia guna memberikan kemudahan bagi pengguna dalam memantau status paket mereka secara instan melalui percakapan. Selain itu, proses \textit{fine-tuning} pada model AI menggunakan data internal perusahaan yang lebih spesifik serta penambahan fitur \textit{payment gateway} untuk transaksi ongkos kirim akan memberikan nilai tambah yang signifikan bagi ekosistem aplikasi. Peningkatan dari sisi keamanan melalui autentikasi dua faktor dan ekspansi ke berbagai kanal pesan instan lainnya juga menjadi langkah penting untuk menjamin perlindungan privasi pengguna sekaligus menjangkau basis pelanggan yang lebih luas secara \textit{multi-channel}.
