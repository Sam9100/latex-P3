% Lampiran

\section*{LAMPIRAN}
\phantomsection
\addcontentsline{toc}{section}{LAMPIRAN}

\subsection*{Lampiran A: Code Coverage}
\phantomsection
\addcontentsline{toc}{subsection}{Lampiran A: Code Coverage}

Berikut adalah detail \textit{code coverage} untuk setiap modul controller pada sistem GOPOS:

\begin{itemize}[leftmargin=*]
    \item \textbf{auth.go} - 69.1\%
    \item \textbf{chat\_history.go} - 58.5\%
    \item \textbf{chatbot.go} - 93.2\%
    \item \textbf{user\_management.go} - 29.1\%
    \item \textbf{whatsauth.go} - 27.3\%
\end{itemize}

\clearpage

\section*{GLOSARIUM}
\phantomsection
\addcontentsline{toc}{section}{GLOSARIUM}

\subsection*{Istilah Teknis}
\phantomsection
\addcontentsline{toc}{subsection}{Istilah Teknis}

\begin{description}[leftmargin=!,labelwidth=4cm]
    \item[Agile (Scrum)] Metodologi pengembangan perangkat lunak yang bersifat iteratif (berulang) dan kolaboratif untuk merespons perubahan dengan cepat.
    
    \item[API] \textit{Application Programming Interface}, antarmuka yang memungkinkan dua aplikasi atau lebih untuk saling berkomunikasi dan bertukar data.
    
    \item[BSON] \textit{Binary JSON}, format penyimpanan data berbasis dokumen yang digunakan oleh MongoDB, mendukung efisiensi ruang dan kecepatan.
    
    \item[Chatbot] Program komputer yang dirancang untuk mensimulasikan percakapan dengan pengguna manusia, terutama melalui internet.
    
    \item[Code Coverage] Metrik untuk mengukur sejauh mana kode sumber program telah dijalankan selama pengujian unit.
    
    \item[Gemini AI] Model bahasa besar (\textit{Large Language Model}) mutakhir dari Google yang mampu memahami dan menghasilkan teks secara natural.
    
    \item[Golang (Go)] Bahasa pemrograman kompilasi yang dikembangkan Google, dikenal karena efisiensi memori dan dukungan konkurensi.
    
    \item[MongoDB] Basis data NoSQL berbasis dokumen yang menyimpan data dalam format fleksibel seperti JSON/BSON.
    
    \item[NLP] \textit{Natural Language Processing}, cabang kecerdasan buatan yang fokus pada interaksi antara komputer dan bahasa manusia agar mesin dapat memahami konteks bicara.
    
    \item[Serverless] Model komputasi di mana pengembang dapat menjalankan kode tanpa harus mengelola infrastruktur server fisik secara manual.
    
    \item[Webhook] Metode komunikasi otomatis di mana satu sistem mengirimkan data ke sistem lain segera setelah suatu peristiwa terjadi.
    
    \item[White Box Testing] Metode pengujian perangkat lunak di mana penguji mengetahui dan menguji struktur internal atau alur kode program.
\end{description}

\subsection*{Istilah Non-Teknis}
\phantomsection
\addcontentsline{toc}{subsection}{Istilah Non-Teknis}

\begin{description}[leftmargin=!,labelwidth=4cm]
    \item[Real-time] Kondisi di mana sistem merespons perintah pengguna secara langsung dan seketika tanpa penundaan waktu yang lama.
    
    \item[User Interface] Tata letak visual dari aplikasi yang memungkinkan pengguna untuk berinteraksi dengan sistem.
    
    \item[Scalability] Kemampuan sistem untuk menangani beban kerja yang meningkat dengan menambah sumber daya secara otomatis.
\end{description}